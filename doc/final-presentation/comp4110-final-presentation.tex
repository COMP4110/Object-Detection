\documentclass{beamer}
\usepackage{graphicx, epstopdf}
\usepackage{amsmath, amssymb, amsbsy, amstext}
% \usepackage{xcolor}

% \definecolor{darkgreen}{rgb}{0,0.6,0}
% \usepackage{pgf}
% \logo{\pgfputat{\pgfxy(-2,6)}{\pgfbox[center,base]{\includegraphics[height=2cm]{amsilogo.jpg}}}}

\usetheme{Darmstadt} 
%\usetheme{Dresden} %Dresden, Darmstadt, Warsaw
% \usecolortheme{dove}
\title{Multiple Instance Boosting for Object Detection}
% \subtitle{(A catchy subtitle)}
\author{Paul Viola, John C. Platt, and Cha Zhang}
\institute{Microsoft Research}
\date{Neural Information Processing Systems, 2005}

\begin{document}

	\maketitle

	\section{Motivation}
	
	\begin{frame}{Object detection}
		\begin{center}
			
			To create an object detector using supervised learning, a correctly
			labelled dataset needs to be created for training.

		\end{center}
	\end{frame}

	
	\begin{frame}{Manual labelling}

		\begin{itemize}
			\item Tedious
			\item Error prone
			\item Ambiguous
		\end{itemize}

		% \begin{center}
			
		% 	This is a tedious and error prone process, as each object
		% 	typically needs to be manually labelled by having a human draw a
		% 	rectangle around it.
			
		% 	The exact position and dimensions that these rectangles should be
		% 	can be ambiguous, as objects may appear in differing orientations
		% 	and deformations.

		% \end{center}
	\end{frame}

	\begin{frame}{Example application}
		\begin{center}
			\includegraphics[width=1\textwidth]{input-1.png} \\
			\includegraphics[width=1\textwidth]{input-2.png}
		\end{center}
	\end{frame}

	\begin{frame}{Example application}

			Harder than face detection:

			\begin{itemize}
				\item People can face away from the camera
				\item Faces are small
				\item Context is required for detection
				\item How much context should be used?
			\end{itemize}

			% This problem is harder than face detection, because people can
			% face away from the camera, and because the faces in the image are
			% small enough that classical face detection algorithms perform
			% inconsistently. We need to use some context (i.e. the surrounding
			% region) to help detect people faces. But how much context to use?
	\end{frame}

	\begin{frame}{Context}
		\begin{center}
			\includegraphics[width=1\textwidth]{windows.png}
		\end{center}
	\end{frame}


	\begin{frame}{Focus of work}
		% \begin{center}

			\begin{itemize}
				\item Object detection is an MIL problem
				\item MIL boost is introduced
				\item Use MIL cost functions in the Anyboost framework
				\item Solve problems associated with manual labelling
			\end{itemize}

			% In this paper, we acknowledge that object detection is innately a MIL
			% problem, and we introduce MIL boost, a boosting method that combines cost
			% functions from MIL literature with the Anyboost framework, to solve the
			% problems associated with manual labelling.
		% \end{center}
	\end{frame}


	\section{Anyboost}


	\begin{frame}{The Anyboost Framework}
		\begin{center}
			\includegraphics[width=1\textwidth]{anyboost-citation.png}

			The anyboost framwork views boosting as iterative gradient
			descent, where a cost functional is minimised over an inner
			product space.

			% \begin{itemize}
			% \end{itemize}
		\end{center}
	\end{frame}

	
	% show a crop of the first page

	% consider showing the table

	\section{MIL}

	\begin{frame}{Multiple Instance Learning}
		\begin{itemize}
			\item Training examples are positive or negative
			\item Group training examples into bags
			\item Bag labels are a logical \(\operatorname{OR}\) of bag contents
			\item Allows the classifier to identify the best training examples in each bag
		\end{itemize}
	\end{frame}


	\section{MIL and Boosting}

	\subsection{Noisy-OR Boost}

	\begin{frame}{Noisy-OR Boost}
		% \begin{center}
			Let $i$ index bags, and $j$ index examples.

			Score of example \(x_{ij}\):
			\[y_{ij} = C(x_{ij})\]
			is a weighted sum of weak classifiers:
			
			\[C(x_{ij}) = \sum_t{\lambda_t c^t(x_{ij})}\]
			
			where \(c(x_{ij}) \in \{-1, +1\}\).

			Define a probability of an example being positive using the logistic function:
			\[p_{ij} = \frac{1}{1 + \operatorname{exp}(-y_{ij})} \in [0, 1] \]

		% \end{center}
	\end{frame}


	\begin{frame}{Bag probability and classifier likelihood}
		% \begin{center}

		Use this to define the probability that a bag is positive using a `noisy OR' % TODO: cite!
		
		
		
		\[p_{i} = 1 - \prod_{j\in i}(1 - p_{ij})\]

		\[p_i = 1 - P(\text{all examples are negative})\]
		\[p_i = P(\text{at least one example is positive})\]


		Define likelihood assigned to a set of training bags.
		(likelihood that the classifier is correct)
		\[L(C) = \prod_{i}{p_i}^{t_i}(1 - p_{i})^(1-t_i)\]
		where \(t_i \in \{0, 1\} \) is the label of bag \(i\).

		% \end{center}
	\end{frame}
	
	\begin{frame}{}
		% \begin{center}

		Weight each example using the derivative of the cost function with respect to the score
		
		\[ \frac{\partial \operatorname{log} L(C)}{\partial y_{ij}} = w_{ij} = \frac{t_i - p_i}{p_i}p_{ij} \]
		
		e.g. if increasing the score increases the probability of the bag being positive, then $w_{ij}$ is positive. Thus, the sign of $w_{ij}$ determines the example label.

		% \end{center}
	\end{frame}
	
	
	\begin{frame}{}
		% \begin{center}

			Each round of boosting is a search for a classifier which maximizes 
			\[\sum_{ij} c(x_{ij})w_{ij}\]

			where \(c(x_{ij}) \in \{-1, +1\}\).
			

			\(\lambda_t\) is determined by to maximize \(\operatorname{log} L(C + \lambda_tc_t)\)

		% \end{center}
	\end{frame}

	\subsection{ISR Boost}

	\begin{frame}{ISR Boost}
		% \begin{center}
			ISR boost was also investigated:

			\(\chi_{ij} = \operatorname{exp}(y_{ij})\), \(S_i = \sum_{j \in i}{\chi_{ij}}\), and \(p_i = \frac{S_i}{1 + S_i}\). \\

			$\chi_{ij}$ can be treated as the likelihood of an object occuring in example $ij$.

			$S_i$ can be interpreted as a likelihood ratio that bag i is positive. \\

			Weights are slightly different
			\[ \frac{\partial \operatorname{log} L(C)}{\partial y_{ij}} = w_{ij} = (t_i - p_i)\frac{\chi_{ij}}{\sum_{j \in i}{\chi_{ij}}} \]
		% \end{center}
	\end{frame}


	\subsection{Comparison of MIL cost functions}

	\begin{frame}{Major differences}
		\begin{center}

			\begin{itemize}
				\item Examples explicitly compete for weight (weak experimental evidence, but might cause only one example in each bag to be labelled positive)

				\item Negative examples also compete for weight. Could cause problems, since negative examples greatly outnumber positive examples
				(Noisy OR criteria treats all negative examples as independent).

			\end{itemize}
		\end{center}
	\end{frame}


	\section{Application}

	\begin{frame}{Detection Cascade}
		\begin{itemize}
			\item Important for speed
			\item Retrain initial classifier after initial training
			\item Repeat for additional performance gains
		\end{itemize}

		% \begin{center}
			% after training the full classifier, retrain the first few weak classifiers to have a very low false negative rate on the examples labeled positive by the full classifier
		% \end{center}
	\end{frame}

	\begin{frame}{Dataset collection}
		% \begin{center}
			\begin{itemize}
				\item Sliding window, between 10000 and 100000 windows in a training image
				\item Each window is an example
				\item 8 videos in different conference rooms
				\item 	1856 images sampled from the set of videos
				\item 	detectors trained on 7 videos, and tested on the remaining one
				\item 	12364 visible people in the images, each manually labeled by drawing a rectangle around the head of each person
			\end{itemize}
		% \end{center}
	\end{frame}	

	\begin{frame}{Positive window and bag generation}
		% \begin{center}
			\begin{itemize}
				\item Windows are automatically labelled based on the manually labelled heads
				\item Tighter bounds are used for Adaboost
				\item One bag for each labelled head, which contains positive windows that overlap the head, and one negative bag for each image
			\end{itemize}
		% \end{center}
	\end{frame}

	\begin{frame}{Positive window and bag generation}
		\begin{center}
			\includegraphics[width=1\textwidth]{windows.png}
		\end{center}
	\end{frame}


	\begin{frame}{Experiment description}
		% \begin{center}
			\begin{itemize}
				\item 	Learning performed on close to 30million subwindows of the 1856 images
				\item 	Monochrome and two `features images' are used
					\begin{itemize}
						\item Difference from the running mean image (something like background subtraction)
						\item Temporal variance over longer time scales
					\end{itemize}
				\item 	set of 2654 rectangle filters are used for training
				\item 	60 filters learned in each experiment. (optimal filter and threshold selected in each round

			\end{itemize}
		% \end{center}
	\end{frame}


	% \begin{frame}{Experiment results}
	% 	\begin{center}
	% 		\begin{itemize}
	% 			\item adaboost compared with both types of MIL boost
	% 		\end{itemize}
	% 	\end{center}
	% \end{frame}


	% \begin{frame}{Definition of positive examples}
	% 	\begin{itemize}
	% 		\item Examples 
	% 		\item for MIL: one bag for each labelled head, which contains positive windows that overlap the head, and one negative bag for each image
	% 	\end{itemize}
	% \end{frame}


	\begin{frame}{Real and corrupted ground truth}
		%\begin{center}
		Two training datasets were tested:
			\begin{itemize}
				\item The original set
				\item Corrupted ground truth (uniform random shift of each box).
			\end{itemize}
		%\end{center}
	\end{frame}


	\begin{frame}{ROC Curves}
		\begin{center}
			\includegraphics[width=1\textwidth]{roc-curves.png}
		\end{center}
	\end{frame}


	\begin{frame}{Typical performance}
		\begin{center}
			\includegraphics[width=1\textwidth]{results-image.png}
		\end{center}
	\end{frame}


	\begin{frame}{Conclusion}
		\begin{center}
			Using MILBoost reduced label noise and improves performance over standard Adaboost.
		\end{center}
	\end{frame}

	
	% \section{Motivation}
	% \begin{frame}{Abstract Strategy Games?} % (2 minutes)
	% 	\begin{definition}
	% 		An \alert{abstract strategy game}, is a two player, \emph{zero sum}
	% 		game with \emph{perfect information}.
	% 	\end{definition}
		
	% 	\begin{itemize}
	% 		\item 
	% 			\emph{Zero sum}: player's losses and gains are perfectly 
	% 			balanced
	% 		\item 
	% 			\emph{Perfect information}: the entire game state is known to 
	% 			all players at all times.
	% 	\end{itemize}
		
	% 	Examples:
	% 	\begin{itemize}
	% 		\item Tic-Tac-Toe
	% 		\item Blackjack
	% 	\end{itemize}
	% \end{frame}
	
	% \begin{frame}{Why study ASGs?} % (1 minute)
	% 	\begin{itemize}
	% 		\item Simple, abstract platform
	% 		\item High complexity (often EXPTIME)
	% 		\item Proficient humans are considered intelligent
	% 	\end{itemize}
	% 	$\therefore$ Computers that can beat humans would need to be intelligent.
	% \end{frame}

	% % \newcommand{\hexBoardImg}{hexBoard.eps}
	% % \newcommand{\hexBoardMath}{
	%  % {\mathchoice
	%   % {\includegraphics[height=1.6ex]{\hexBoardImg}}
	%   % {\includegraphics[height=1.6ex]{\hexBoardImg}}
	%   % {\includegraphics[height=1.2ex]{\hexBoardImg}}
	%   % {\includegraphics[height=0.9ex]{\hexBoardImg}}
	%  % }
	% % }

	% \subsection{Example} 
	% \begin{frame}{Tree search} % don't spend long at all on this...
	% 	(insert magical animated picture here)
	% \end{frame}
	% \begin{frame}{Evaluation Function}
	% 	% $$ \operatorname{Utility}(\hexBoardImg) $$
	% 	\resizebox{0.8\linewidth}{!}{ \ensuremath{\mu\left(\vcenter{\hbox{\includegraphics[height=3ex]{hex_blueWin.eps}}} \right) = \infty }}
	% 	\resizebox{0.8\linewidth}{!}{ \ensuremath{\mu\left(\vcenter{\hbox{\includegraphics[height=3ex]{hex_redWin.eps}}} \right) = -\infty }}
	% \end{frame}
	% \begin{frame}{Transposition Tables}
	% 	\includegraphics[width=0.395\linewidth]{hex_tt_1_f.eps}
	% 	\includegraphics[width=0.395\linewidth]{hex_tt_1_t.eps}
	% \end{frame}
	% \begin{frame}{Transposition Tables}
	% 	\includegraphics[width=0.395\linewidth]{hex_tt_1_f.eps}
	% 	\includegraphics[width=0.395\linewidth]{hex_tt_1_t.eps}
	% \end{frame}
	
	% \begin{frame}{Human Psychology}
	% 	\begin{itemize}
	% 		\item There are many potential explanations for the AI effect
	% 		\item One is based around preservation of human ego
	% 		\item Humanity is unique on this planet in our intelligence most of all, and we subconsciously preserve this importance by discarding AI attempts
	% 		\item Another involves the mysterious nature of intelligence – once we understand the process it is no longer mysterious and cannot therefore be intelligence by our internal definition
	% 	\end{itemize}
	% 	\includegraphics[width=0.5\linewidth]{ai1.png}
	% \end{frame}

	
\end{document}

