\documentclass{beamer}
\usepackage{graphicx, epstopdf}
\usepackage{amsmath, amssymb, amsbsy, amstext}
% \usepackage{xcolor}

\usepackage[utf8]{inputenc}
\usepackage{natbib}

\usepackage{todonotes}
% \usepackage[disable]{todonotes}
\presetkeys{todonotes}{inline}{}

\usepackage{booktabs} % for much better looking tables
\usepackage{tabularx}

\newcommand{\code}[1]{{\texttt{#1}}}
\newcommand{\libraryname}[1]{{\texttt{#1}}}
\newcommand{\codefile}[1]{{\textit{#1}}}
\newcommand{\program}[1]{\code{#1}}
\newcommand{\taskname}[1]{{\textit{#1}}}
\newcommand{\newterm}[1]{{\textit{#1}}}
\newcommand{\scarequotes}[1]{`#1'}

% \definecolor{darkgreen}{rgb}{0,0.6,0}
% \usepackage{pgf}
% \logo{\pgfputat{\pgfxy(-2,6)}{\pgfbox[center,base]{\includegraphics[height=2cm]{amsilogo.jpg}}}}

\usetheme{Darmstadt} 
%\usetheme{Dresden} %Dresden, Darmstadt, Warsaw
% \usecolortheme{dove}
% \title{Sphere Detection using Boosted Classifiers}
\title{Comparing Feature Desctiptor Performance of Boosted Classifiers }
% \subtitle{Computer Vision Interim Report}
\author{Brendan Annable, Mitchell Metcalfe, Monica Olejniczak}
\institute{The University of Newcastle, Australia}
\date{\today}

\begin{document}

	\maketitle

	
	% \begin{frame}{Object detection}
	% 	\begin{center}
	% 		To create an object detector using supervised learning, a correctly
	% 		labelled dataset needs to be created for training.
	% 	\end{center}
	% \end{frame}

\section{Introduction}

	\subsection{Motivation}

	\begin{frame}{Robot soccer}
		Robots at RoboCup \citep{KitanoAKNO97} need to know where the ball is.

		Basic approach is circle detection, which has problems:
		\begin{itemize}
			\item centre circle
	        \item penalty spots
	        \item line intersections
	        \item anything that looks round from some angle
		\end{itemize}
	\end{frame}

	\begin{frame}{Current approaches}
		A review of recent literature reveals several recent approaches:
		\begin{itemize}
			\item centre circle
	        \item penalty spots
	        \item line intersections
	        \item anything that looks round from some angle
		\end{itemize}
	\end{frame}

	\begin{frame}{Sphere detection problem}
		\begin{itemize}
			\item More general problem than ball detction.
			\item Detect spheres.
			\item Do not detect any \scarequotes{round things} unless they are spheres.
		\end{itemize}
	\end{frame}

\section{Focus of work}
	\begin{frame}{Proposed approach}
		Attempt to solve the more generic problem of sphere detection using a boosted classifier cascade.
	\end{frame}

	\begin{frame}{Feature comparison}
		Compare appropriateness of the following feature types for the problem:
		\begin{itemize}
			\item Extended
        Haar features \citep{Lienhart2002extended};
			\item Local Binary Patterns (LBPs) \citep{liao2007learning}
			\item Histograms of Oriented Gradients (HoG) features \citep{dalal2005histograms}
		\end{itemize}
	\end{frame}

\section{Methodology}
	% 
	%  (describe the goal of comparing techniques)

\section{Training}

\begin{frame}{Dataset}
	All training and testing images were sourced from ImageNet \citep{imagenet_cvpr09}.
\end{frame}

\begin{frame}{Positive samples}
	\begin{table}[H]
		\footnotesize
		\centering
		\caption{ImageNet synsets used for the positive samples in the training set and their associated WordNet \citep{fellbaum1998wordnet} identifiers.}
		\label{tab:postraining}
		\begin{tabularx}{\textwidth}{lX}
			\toprule
			\textbf{Id} & \textbf{Category} \\
			\midrule
			n02799071 & Baseball \\
			n02802426 & Basketball \\
			n02882301 & Bowling ball, bowl \\
			n03134739 & Croquet ball \\
			n03145719 & Cue ball \\
			n03267113 & Eight ball \\
			n03721047 & Marble \\
			n03742019 & Medicine ball \\
			n03825442 & Ninepin ball, skittle ball \\
			n03942813 & Ping-pong ball \\
			n03982232 & Pool ball \\
			n04409515 & Tennis ball \\
			n04540053 & Volleyball \\
			\bottomrule
		\end{tabularx}
	\end{table}
\end{frame}

\begin{frame}{Positive samples (\scarequotes(Simple))}
	\begin{table}[H]
		\footnotesize
		\centering
		\caption{ImageNet synsets used for the positive samples in the training set and their associated WordNet \citep{fellbaum1998wordnet} identifiers.}
		\label{tab:postraining}
		\begin{tabularx}{\textwidth}{lX}
			\toprule
			\textbf{Id} & \textbf{Category} \\
			\midrule
			\textcolor{red}{n02799071} & \textcolor{red}{Baseball} \\
			n02802426 & Basketball \\
			n02882301 & Bowling ball, bowl \\
			\textcolor{red}{n03134739} & \textcolor{red}{Croquet ball} \\
			\textcolor{red}{n03145719} & \textcolor{red}{Cue ball} \\
			n03267113 & Eight ball \\
			n03721047 & Marble \\
			n03742019 & Medicine ball \\
			\textcolor{red}{n03825442} & \textcolor{red}{Ninepin ball, skittle ball} \\
			\textcolor{red}{n03942813} & \textcolor{red}{Ping-pong ball} \\
			\textcolor{red}{n03982232} & \textcolor{red}{Pool ball} \\
			n04409515 & Tennis ball \\
			n04540053 & Volleyball \\
			\bottomrule
		\end{tabularx}
	\end{table}
\end{frame}

\begin{frame}{Test set}
	\begin{table}[H]
		\centering
		\caption{Additional WordNet identifiers for positive samples in the test set.}
		\label{tab:postest}
		\begin{tabularx}{\textwidth}{lX}
			\toprule
			\textbf{Id} & \textbf{Category} \\
			\midrule
				n02778669 & Ball \\
				n02779435 & Ball \\
				n02839351 & Billiard ball \\
				n02778669 & Generic sporting equipment balls \\
				n03445777 & Golf ball \\
				n02779435 & Plaything, toy, ball \\
				n04254680 & Soccer ball \\
			\bottomrule
		\end{tabularx}
	\end{table}
\end{frame}

\begin{frame}{Negative samples}
	\begin{table}[H]
		\centering
		\caption{WordNet identifiers for hard negative samples in the training set.}
		\label{tab:negtraining}
		\begin{tabularx}{\textwidth}{lX}
			\toprule
			\textbf{Id} & \textbf{Category} \\
			\midrule
				n03032811 & Circle, round \\
				n13873917 & Circle \\
				n13873502 & Circle \\
			\bottomrule
		\end{tabularx}
	\end{table}
\end{frame}

\begin{frame}{Background samples}
	\begin{table}[H]
		\centering
		\caption{WordNet identifiers for background samples in the test set.}
		\label{tab:baktraining}
		\begin{tabularx}{\textwidth}{lX}
			\toprule
			\textbf{Id} & \textbf{Category} \\
			\midrule
				n02782778 & Ballpark, park \\
				n08659446 & Field \\
				n03841666 & Office, business office \\
			\bottomrule
		\end{tabularx}
	\end{table}
\end{frame}

\begin{frame}{Software used}
	\begin{itemize}
		\item Classifiers of each feature type were trained simultaneously
		\item Run on a Macbook Pro with an Intel Core i7 (I7-3740QM) processor and 16GB RAM.
	\end{itemize}
\end{frame}


\section{Results}

% Make a table of results
% Hopefully a graph?

\section{Conclusion}

	\subsection{Feature performance comparison}
	%   - Haar features clearly performed better than HOGS or LBP on the sphere detection task
	%   - 

	\subsection{Future work}
	% Future Work
	%   - 

	\subsection{Bibliography}

	\begin{frame}{Bibliography}
		\bibliography{references}
		\bibliographystyle{apalike}
	\end{frame}

\end{document}

