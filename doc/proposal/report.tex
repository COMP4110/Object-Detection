% !TEX TS-program = pdflatex
% !TEX encoding = UTF-8 Unicode

% This is a simple template for a LaTeX document using the "article" class.
% See "book", "report", "letter" for other types of document.

\documentclass[11pt]{scrartcl} % use larger type; default would be 10pt

\usepackage[utf8]{inputenc} % set input encoding (not needed with XeLaTeX)
\usepackage[title,titletoc,header]{appendix}
\usepackage{multicol}
% \usepackage{titling}
\usepackage{longtable}

\usepackage{amsmath}
\usepackage{hyperref}

%%% Examples of Article customizations
% These packages are optional, depending whether you want the features they provide.
% See the LaTeX Companion or other references for full information.

%%% PAGE DIMENSIONS
\usepackage[a4paper]{geometry} % to change the page dimensions
% \geometry{a4paper, margin=1.3in} % or letterpaper (US) or a5paper or....
% \geometry{margin=2in} % for example, change the margins to 2 inches all round
% \geometry{landscape} % set up the page for landscape
%   read geometry.pdf for detailed page layout information

\usepackage{graphicx} % support the \includegraphics command and options

% \usepackage[parfill]{parskip} % Activate to begin paragraphs with an empty line rather than an indent

%%% PACKAGES
\usepackage{booktabs} % for much better looking tables
\usepackage{array} % for better arrays (eg matrices) in maths
\usepackage{paralist} % very flexible & customisable lists (eg. enumerate/itemize, etc.)
\usepackage{verbatim} % adds environment for commenting out blocks of text & for better verbatim
\usepackage{subfig} % make it possible to include more than one captioned figure/table in a single float
% These packages are all incorporated in the memoir class to one degree or another...

\usepackage{pgfgantt} % gantt charts
% \usepackage[export]{adjustbox}[2011/08/13] % For centering wide figures

%%% HEADERS & FOOTERS
\usepackage{fancyhdr} % This should be set AFTER setting up the page geometry
\pagestyle{fancy} % options: empty , plain , fancy
\renewcommand{\headrulewidth}{0pt} % customise the layout...
\lhead{}\chead{}\rhead{}
\lfoot{}\cfoot{\thepage}\rfoot{}

%%% SECTION TITLE APPEARANCE
\usepackage{sectsty}
\allsectionsfont{\sffamily\mdseries\upshape} % (See the fntguide.pdf for font help)
% (This matches ConTeXt defaults)

%%% ToC (table of contents) APPEARANCE
\usepackage[nottoc,notlof,notlot]{tocbibind} % Put the bibliography in the ToC
\usepackage[titles,subfigure]{tocloft} % Alter the style of the Table of Contents
\renewcommand{\cftsecfont}{\rmfamily\mdseries\upshape}
\renewcommand{\cftsecpagefont}{\rmfamily\mdseries\upshape} % No bold!


\newcommand{\code}[1]{{\texttt{#1}}}
\newcommand{\libraryname}[1]{{\texttt{#1}}}
\newcommand{\codefile}[1]{{\textit{#1}}}
\newcommand{\program}[1]{\code{#1}}
\newcommand{\taskname}[1]{{\textit{#1}}}

%%% END Article customizations

% \setlength{\parindent}{0pt}
% \setlength{\parskip}{2ex plus 0.5ex minus 0.3ex}

%%% The "real" document content comes below...

% TODO: Catchy project title
\title{COMP4110 Project Proposal}
\subtitle{Fast detection of arbitrary balls for robot soccer}
% \subtitle{A 3D testing environment for an enhanced virtual reality system}
\author{ Brendan Annable, Mitchell Metcalfe, Monica Olejniczak }

\date{\today} % Activate to display a given date or no date (if empty),
              % otherwise the current date is printed
\rhead{ COMP4110, Project Proposal, \today }

\begin{document}
% \newgeometry{top=2cm}
\maketitle
% \vspace{-1.5 cm}
% \restoregeometry

\begin{abstract}
    
    
    
\end{abstract}

% \newpage
\tableofcontents
% \newpage

\section{Background}


\section{Project Description}


\section{Project Schedule}

    \begin{figure}[H]
        \makebox[\textwidth][c]{\resizebox{0.95\paperwidth}{!}{% Define some Gantt chart helper commands:
\newcommand{\completedganttbar}[4][]{ %
    \ganttbar[bar/.append style={draw=gray, fill=gray},#1]{#2}{#3}{#4} %
}
\newcommand{\optionalganttbar}[4][]{ %
    \ganttbar[bar/.append style={draw=gray, pattern color=gray, pattern=north west lines},#1]{#2}{#3}{#4} %
}
\newcommand{\optionalganttlinkedbar}[4][]{ %
    \ganttlinkedbar[bar/.append style={draw=gray, pattern color=gray, pattern=north west lines},#1]{#2}{#3}{#4} %
}

\begin{ganttchart}[
        hgrid,
        vgrid={*6{black, dotted},*1{black, dashed}}, % Note: NO SPACES!
        title height = 1,
        x unit=0.3cm,
        y unit title=0.75cm,
        y unit chart=1cm,
        time slot format=isodate,
        % progress=today,
        % today=2015-8-20,
        % bar/.append style={fill=green},
        % bar incomplete/.append style={fill=white},
        % group incomplete/.append style={draw=black,fill=none},
        % progress label text={}
        ]
        {2015-03-16} % start date: Mon, Mar-16. Week 4.
        {2015-05-31} % end date: Sun, May-31.
\setganttlinklabel{f-s}{}

% Gantt chart header:
% Note: Week starts on Sunday.
\gantttitlecalendar*{2015-03-16}{2015-05-31}{month=name} \\
\gantttitlecalendar*{2015-03-16}{2015-04-05}{week=4}
\gantttitle{Semester 1 Recess}{14}
\gantttitlecalendar*{2015-04-20}{2015-05-31}{week=7}

% Proposal
\ganttnewline \ganttgroup{Project proposal}{2015-03-19}{2015-03-26}
\ganttnewline \ganttbar{Report}{2015-03-19}{2015-03-23}
\ganttnewline \ganttbar{Presentation}{2015-03-24}{2015-03-26}

% Project work
\ganttnewline \ganttgroup{Project work}{2015-03-27}{2015-04-30}
\ganttnewline \ganttbar{Dataset collection}{2015-03-27}{2015-04-05}
\ganttnewline \ganttbar{Candidate selection}{2015-04-01}{2015-04-08}
\ganttnewline \ganttbar{Development}{2015-04-09}{2015-04-22}
\ganttnewline \ganttbar{Report preparation}{2015-04-23}{2015-04-30}

% Final report
\ganttnewline \ganttgroup{Final report}{2015-04-31}{2015-05-28}
\ganttnewline \ganttbar{Detection performance testing}{2015-04-31}{2015-05-07}
\ganttnewline \ganttbar{Analysis of results}{2015-05-08}{2015-05-15}
\ganttnewline \ganttbar{Report preparation}{2015-05-16}{2015-05-28}

% % Examples of gantt chart capabilities: 
% \ganttnewline \ganttgroup{Label Text}{2015-08-20}{2015-10-5}
% \ganttnewline \ganttmilestone{Label Text}{2015-08-3}
% \ganttnewline \completedganttbar{Label Text}{2015-8-9}{2015-8-13}
% \ganttnewline \ganttlinkedmilestone{Label Text}{2015-08-19}
% \ganttnewline \ganttbar{Label Text}{2015-09-11}{2015-10-5}
% \ganttnewline \optionalganttbar{Label Text}{2015-10-6}{2015-10-26}

\end{ganttchart}
}}
        \caption[Project Schedule]{
            A Gantt chart illustrating the planned project schedule.
            Patterned grey bars represent optional tasks.
        }
        \label{gantt:proposal}
    \end{figure}

\section{Summary}


\end{document}

