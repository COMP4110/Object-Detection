% !TEX TS-program = pdflatex
% !TEX encoding = UTF-8 Unicode

% This is a simple template for a LaTeX document using the "article" class.
% See "book", "report", "letter" for other types of document.

\documentclass[11pt]{scrartcl} % use larger type; default would be 10pt

\usepackage[utf8]{inputenc} % set input encoding (not needed with XeLaTeX)
\usepackage[title,titletoc,header]{appendix}
\usepackage{multicol}
% \usepackage{titling}
\usepackage{longtable}

\usepackage{amsmath}
\usepackage{hyperref}

\usepackage{todonotes}
\presetkeys{todonotes}{inline}{}

\usepackage{natbib}
% To cite, use \citet{} in text citation, or \citep{} (in parentheses).

%%% PAGE DIMENSIONS
\usepackage[a4paper]{geometry} % to change the page dimensions
% \geometry{a4paper, margin=1.3in} % or letterpaper (US) or a5paper or....
% \geometry{margin=2in} % for example, change the margins to 2 inches all round
% \geometry{landscape} % set up the page for landscape
%   read geometry.pdf for detailed page layout information

\usepackage{graphicx} 				% support the \includegraphics command and options

% \usepackage[parfill]{parskip} % Activate to begin paragraphs with an empty line rather than an indent

%%% PACKAGES
\usepackage{booktabs} 				% for much better looking tables
\usepackage{array} 					% for better arrays (eg matrices) in maths
\usepackage{paralist} 				% very flexible & customisable lists (eg. enumerate/itemize, etc.)
\usepackage{verbatim} 				% adds environment for commenting out blocks of text & for better verbatim
\usepackage{subfig} 				% make it possible to include more than one captioned figure/table in a single float
\usepackage{float} 					% allow floating for figures

\usepackage{pgfgantt} 				% gantt charts
% \usepackage[export]{adjustbox}[2011/08/13] % For centering wide figures

%%% PARAGRAPHING & INDENTATION
\setlength{\parindent}{0pt}
\setlength{\parskip}{2ex plus 0.5ex minus 0.3ex}

%%% HEADERS & FOOTERS
\usepackage{fancyhdr} 				% This should be set AFTER setting up the page geometry
\pagestyle{fancy} 					% options: empty , plain , fancy
\renewcommand{\headrulewidth}{0pt} 	% customise the layout...
\lhead{}\chead{}\rhead{}
\lfoot{}\cfoot{\thepage}\rfoot{}

%%% SECTION TITLE APPEARANCE
\usepackage{sectsty}
\allsectionsfont{\sffamily\mdseries\upshape} % (See the fntguide.pdf for font help)
% (This matches ConTeXt defaults)

%%% TABLE OF CONTENTS APPEARANCE
 % Put the bibliography in the ToC
\usepackage[nottoc,notlof,notlot]{tocbibind}
 % Alter the style of the Table of Contents
\usepackage[titles,subfigure]{tocloft}
\renewcommand{\cftsecfont}{\rmfamily\mdseries\upshape}
 % No bold!
\renewcommand{\cftsecpagefont}{\rmfamily\mdseries\upshape}

\newcommand{\code}[1]{{\texttt{#1}}}
\newcommand{\libraryname}[1]{{\texttt{#1}}}
\newcommand{\codefile}[1]{{\textit{#1}}}
\newcommand{\program}[1]{\code{#1}}
\newcommand{\taskname}[1]{{\textit{#1}}}

% TODO: Catchy project title
\title{COMP4110 Project Proposal}
\subtitle{Fast detection of arbitrary balls for robot soccer}
% \subtitle{A 3D testing environment for an enhanced virtual reality system}
\author{ Brendan Annable, Mitchell Metcalfe, Monica Olejniczak }

% Activate to display a given date or no date (if empty), otherwise the current date is printed
\date{\today}

\rhead{COMP4110, Project Proposal, \today}

\begin{document}
	% \newgeometry{top=2cm}
	\maketitle
	% \vspace{-1.5 cm}
	% \restoregeometry

	\begin{abstract}
		
		
	\end{abstract}

	\newpage
	\tableofcontents
	\newpage


	\section{Summary} {

        % This project aims to detect spheres within an image, based on the
        % shading around its environment, and will have a heavy focus on
        % performance and accuracy. The results of this project will be utilised
        % by the Newcastle Robotics Laboratory and has the potential to impact
        % their results at RoboCup, an international robotic soccer competition.
        % This project is of significant interest to the robot soccer community
        % as the results of this could yield a more efficient method of real-
        % time ball detection, which is also beneficial to those who are
        % interested in ball detection within a competitive sports environment
        % such as basketball or tennis.
		
	}

	\section{Background and aims} {

  %       This project aims to develop methods for sphere detection in a RoboCup
  %       environment, placing an emphasis on performance due to the limited
  %       capabilities of the target platform. In particular, the performance is
  %       significantly effected by the number of pixel samples read from a
  %       given image and will be minimised. The project will focus on
  %       developing a novel, minimal, multi-stage scanning and ball-detection
  %       strategy that minimises the sampling effort while still correctly
  %       recognising all balls that are present in the image without false
  %       positives.

  %       An implementation of the developed approach will be created by
  %       extending the NUbots’ existing vision system, to minimise the effort
  %       required. The results will be compared to the ball detection method
  %       used in their current vision system and to other competing ball
  %       detection techniques, with a heavy focus on performance and accuracy.
  %       If a robust approach is found, it could be generalised to detect other
  %       shapes such as robots, lines and goals based on a set of features.

  %       It is also possible to utilise the information gathered from previous
  %       frames, or from the robot platform itself, including the expected
  %       position, velocity and kinematics horizon.

        Keeping track of the ball is so natural and fundamental an aspect of
        playing soccer that many human players would hardly think it to be a
        skill in itself.
        However, while it may come naturally to humans, fast and reliable ball
        tracking has presented a challenge that has attracted much research in
        the robot soccer community.
        
        \todo { 
            Refer to the progression of ball detection techniques from colour based to
            shape based and intensity based.

            Describe our approach as a natural
            next step.

            Add references here.
        }

        Early attempts at ball detection at Robocup simply used histogramming
        techniques targeted at a specific range of colour intensities to find
        the orange ball.  As the Robocup playing field became less structured,
        teams have increasingly implemented methods that detect the shape of
        the ball as well.
        \citet{schulz2007ball} used a neural network on
        subsampled luminance images of the ball to detect the ball's shape.
        Recent approaches have focused on detecting the approximately
        circular shape of the ball in typical images. These include
        clustering, hough filters, and RANSAC.

        \todo {
            Make a note about colour classification: 

            "The preprocessing step of colour classification using a precalculated
            lookup table has become standard in the Standard Platform and Kid Size
            leagues."
        }

        Many current ball detection methods make assumptions about the ball or
        the environment that limit their applicability elsewhere. Methods
        based on colour classification or similarity can suffer from false
        positives due to other objects of similar colours, and can suffer from
        both false positives and false negatives due to unexpected changes in
        lighting. Methods based on circle or ellipse detection can suffer from
        false positives due to the presence of disk-like objects in the
        environment, or due to objects that appear circular when viewed from
        specific angles.

        \todo {
            Posit the circular appearance of disk-like objects as the worst
            case for current ball-detection methods.
        }

        \todo {
            % This may be unrelated enough that we can just not mention it.

            Note the success of trajectory estimation in helping rule out
            false positives in ball detection for automated annotation of
            sports footage, but indicate that this technique alone is not
            enough to filter out all (non-spherical) false positives (e.g. a
            stationary disk will follow a valid trajectory for a ball).

			Indicate that we would like an algorithm that also works on single images.

            Also note that some methods attempt to gain accuracy by attempting
            only to recognise a specific ball (e.g. the paper that used patch
            descriptors to try and match the pattern on a given soccer ball).
        }

        \todo {
            Hypothesise that 3D features of a sphere should be enough to
            distinguish them from other round objects.

            (for the common, but important case of spheres resting on a ground
            plane, that are primarily lit from above.)
        }

        To avoid the limitations 

		For simplicity, the scope of the proposed project is limited
        to the common and important case of detecting balls that are resting
        on the ground

        \todo { 
            Hypothesise that spatial relationships between separate 3D features (e.g.
            specular highlight, shadow, shading gradient, the projected ellipse)
            should help to identify spheres, as they have helped in detecting faces.
        }

        % % Note: Let's not do this. It's a bit too strong/unlikely to work.
        % %       We should instead just mention patch descriptors as a possible
        % %       technique in the project plan?
        % \todo {
        %     Hypothesise that local feature/patch descriptors should be enough
        %     to identify the specular highlight and shadow cast by a sphere.
        % }

        \todo { 
            Contrast the goal of our proposed project with the focus of most other
            algorithms, which generally do not attempt to verify that they have not
            erroneously detected disks.
        }

        As a result of targeting our approach toward the 3D features that
        distinguish spheres from objects, such as disks, we expect our method
        to be particularly robust to detecting false positives.

	}

	\section{Project plan} {

        \todo {
            Outline the tendency of ball and face detection approaches to use a
            coarse/cheap candidate detection pass and a more accurate/costly
            verification/outlier rejection pass, and propose that we follow the
            same approach and investigate one or two alternatives (Hough
            transform, \citet{Yuan2015} or \citet{Pan2011}, a more ad-hoc
            method?) and compare results - possibly by leveraging the NUbots'
            existing system.

            % Note: we could just use 'cascaded classifiers' 
        }

        \todo {
            Propose that we evaluate and compare a number of methods for the
            verification stage (i.e. the stage that differentiates between spheres
            and disks), and list at least two methods for doing so (SVMs, CNNs,
            KNNs, Decision forests, Adaboost?). Indicate that we should
            investigate simple input representations/transformations that might
            help to make these methods more effective (e.g. Fourier transforms,
            wavelets, local binary descriptors, patch descriptors, Haar-like
            features).
        }

        \todo {
            Outline how the methods proposed will be evaluated and compared, with
            reference to any expected difficulties.
        }

        \subsection{Image dataset creation} {

            \todo {
                Propose that we build our own dataset of images of spheres
                (and disks) in varied lighting conditions and environments (to
                avoid having to purchase images or look into
                copyright/reproduction concerns). Describe how we will obtain
                a collection of spheres, what the collection might contain,
                how we will select environments and lighting conditions, and
                how we will take the photos (include a specification of the
                camera + image quality that will be used).
            }

            \todo {
                Indicate whether/how we will label the dataset
            }

            \todo {
                Indicate whether the dataset will contain images of moving balls
            }

            \todo {
                Indicate how we might augment the dataset with transformed
                images, etc.
            }

            \todo {
                Indicate whether images of non spherical objects will appear
                in the dataset, which objects will be selected for inclusion,
                and how the images will be obtained.
            }

            \todo {
                Use the robot soccer field as one of the environments
            }

            \todo {
                Propose that (aspheric) deflated balls be imaged in addition
                to (spherical) inflated balls
            }
        }

	}

	\section{Project schedule} {

        \todo {
            Propose a timeline containing presentation preparation, dataset
            collection, candidate selection development, classifier development,
            detection performance testing, analysis of results, report
            preparation, and final presentation preparation.
        }

		\begin{figure}[H]
	        \makebox[\textwidth][c]{\resizebox{0.95\paperwidth}{!}{% Define some Gantt chart helper commands:
\newcommand{\completedganttbar}[4][]{ %
    \ganttbar[bar/.append style={draw=gray, fill=gray},#1]{#2}{#3}{#4} %
}
\newcommand{\optionalganttbar}[4][]{ %
    \ganttbar[bar/.append style={draw=gray, pattern color=gray, pattern=north west lines},#1]{#2}{#3}{#4} %
}
\newcommand{\optionalganttlinkedbar}[4][]{ %
    \ganttlinkedbar[bar/.append style={draw=gray, pattern color=gray, pattern=north west lines},#1]{#2}{#3}{#4} %
}

\begin{ganttchart}[
        hgrid,
        vgrid={*6{black, dotted},*1{black, dashed}}, % Note: NO SPACES!
        title height = 1,
        x unit=0.3cm,
        y unit title=0.75cm,
        y unit chart=1cm,
        time slot format=isodate,
        % progress=today,
        % today=2015-8-20,
        % bar/.append style={fill=green},
        % bar incomplete/.append style={fill=white},
        % group incomplete/.append style={draw=black,fill=none},
        % progress label text={}
        ]
        {2015-03-16} % start date: Mon, Mar-16. Week 4.
        {2015-05-31} % end date: Sun, May-31.
\setganttlinklabel{f-s}{}

% Gantt chart header:
% Note: Week starts on Sunday.
\gantttitlecalendar*{2015-03-16}{2015-05-31}{month=name} \\
\gantttitlecalendar*{2015-03-16}{2015-04-05}{week=4}
\gantttitle{Semester 1 Recess}{14}
\gantttitlecalendar*{2015-04-20}{2015-05-31}{week=7}

% Proposal
\ganttnewline \ganttgroup{Project proposal}{2015-03-19}{2015-03-26}
\ganttnewline \ganttbar{Report}{2015-03-19}{2015-03-23}
\ganttnewline \ganttbar{Presentation}{2015-03-24}{2015-03-26}

% Project work
\ganttnewline \ganttgroup{Project work}{2015-03-27}{2015-04-30}
\ganttnewline \ganttbar{Dataset collection}{2015-03-27}{2015-04-05}
\ganttnewline \ganttbar{Candidate selection}{2015-04-01}{2015-04-08}
\ganttnewline \ganttbar{Development}{2015-04-09}{2015-04-22}
\ganttnewline \ganttbar{Report preparation}{2015-04-23}{2015-04-30}

% Final report
\ganttnewline \ganttgroup{Final report}{2015-04-31}{2015-05-28}
\ganttnewline \ganttbar{Detection performance testing}{2015-04-31}{2015-05-07}
\ganttnewline \ganttbar{Analysis of results}{2015-05-08}{2015-05-15}
\ganttnewline \ganttbar{Report preparation}{2015-05-16}{2015-05-28}

% % Examples of gantt chart capabilities: 
% \ganttnewline \ganttgroup{Label Text}{2015-08-20}{2015-10-5}
% \ganttnewline \ganttmilestone{Label Text}{2015-08-3}
% \ganttnewline \completedganttbar{Label Text}{2015-8-9}{2015-8-13}
% \ganttnewline \ganttlinkedmilestone{Label Text}{2015-08-19}
% \ganttnewline \ganttbar{Label Text}{2015-09-11}{2015-10-5}
% \ganttnewline \optionalganttbar{Label Text}{2015-10-6}{2015-10-26}

\end{ganttchart}
}}
			\caption[Project Schedule] {
                A Gantt chart illustrating the planned project schedule.
                Patterned grey bars represent optional tasks.
                \todo {
                    Fill in the Gantt chart
                }
			}
			\label{gantt:proposal}
		\end{figure}

	}

	\section{Roles} {
        \todo {
            Propose that each team member works on a different candidate detection
            method, and a different verification method, and that all combinations
            are then tested on the dataset.
        }



    }

    \section{Communication of results} {
        The results of the project will be compiled into a report (as outlined
        in the project schedule) that is intended to be suitable for
        submission to a relevant conference or journal. In the case that the
        report is accepted for publication, the dataset created for the
        project will also be made available online.
    }

	\bibliography{bibliography}
	\bibliographystyle{apalike}

\end{document}

