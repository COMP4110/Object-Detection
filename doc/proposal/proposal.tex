% !TEX TS-program = pdflatex
% !TEX encoding = UTF-8 Unicode

% This is a simple template for a LaTeX document using the "article" class.
% See "book", "report", "letter" for other types of document.

\documentclass[11pt]{scrartcl} % use larger type; default would be 10pt

\usepackage[utf8]{inputenc} % set input encoding (not needed with XeLaTeX)
\usepackage[title,titletoc,header]{appendix}
\usepackage{multicol}
% \usepackage{titling}
\usepackage{longtable}

\usepackage{amsmath}
\usepackage{hyperref}

\usepackage{natbib}
% To cite, use \citet{} in text citation, or \citep{} (in parentheses).

%%% PAGE DIMENSIONS
\usepackage[a4paper]{geometry} % to change the page dimensions
% \geometry{a4paper, margin=1.3in} % or letterpaper (US) or a5paper or....
% \geometry{margin=2in} % for example, change the margins to 2 inches all round
% \geometry{landscape} % set up the page for landscape
%   read geometry.pdf for detailed page layout information

\usepackage{graphicx} 				% support the \includegraphics command and options

% \usepackage[parfill]{parskip} % Activate to begin paragraphs with an empty line rather than an indent

%%% PACKAGES
\usepackage{booktabs} 				% for much better looking tables
\usepackage{array} 					% for better arrays (eg matrices) in maths
\usepackage{paralist} 				% very flexible & customisable lists (eg. enumerate/itemize, etc.)
\usepackage{verbatim} 				% adds environment for commenting out blocks of text & for better verbatim
\usepackage{subfig} 				% make it possible to include more than one captioned figure/table in a single float
\usepackage{float} 					% allow floating for figures

\usepackage{pgfgantt} 				% gantt charts
% \usepackage[export]{adjustbox}[2011/08/13] % For centering wide figures

%%% PARAGRAPHING & INDENTATION
\setlength{\parindent}{0pt}
\setlength{\parskip}{2ex plus 0.5ex minus 0.3ex}

%%% HEADERS & FOOTERS
\usepackage{fancyhdr} 				% This should be set AFTER setting up the page geometry
\pagestyle{fancy} 					% options: empty , plain , fancy
\renewcommand{\headrulewidth}{0pt} 	% customise the layout...
\lhead{}\chead{}\rhead{}
\lfoot{}\cfoot{\thepage}\rfoot{}

%%% SECTION TITLE APPEARANCE
\usepackage{sectsty}
\allsectionsfont{\sffamily\mdseries\upshape} % (See the fntguide.pdf for font help)
% (This matches ConTeXt defaults)

%%% TABLE OF CONTENTS APPEARANCE
 % Put the bibliography in the ToC
\usepackage[nottoc,notlof,notlot]{tocbibind}
 % Alter the style of the Table of Contents
\usepackage[titles,subfigure]{tocloft}
\renewcommand{\cftsecfont}{\rmfamily\mdseries\upshape}
 % No bold!
\renewcommand{\cftsecpagefont}{\rmfamily\mdseries\upshape}

\newcommand{\code}[1]{{\texttt{#1}}}
\newcommand{\libraryname}[1]{{\texttt{#1}}}
\newcommand{\codefile}[1]{{\textit{#1}}}
\newcommand{\program}[1]{\code{#1}}
\newcommand{\taskname}[1]{{\textit{#1}}}

% TODO: Catchy project title
\title{COMP4110 Project Proposal}
\subtitle{Fast detection of arbitrary balls for robot soccer}
% \subtitle{A 3D testing environment for an enhanced virtual reality system}
\author{ Brendan Annable, Mitchell Metcalfe, Monica Olejniczak }

% Activate to display a given date or no date (if empty), otherwise the current date is printed
\date{\today}

\rhead{COMP4110, Project Proposal, \today}

\begin{document}
	% \newgeometry{top=2cm}
	\maketitle
	% \vspace{-1.5 cm}
	% \restoregeometry

	\begin{abstract}
		
		This is a cool test where we cite \citep{Yuan2015}.
		
	\end{abstract}

	\newpage
	\tableofcontents
	\newpage

	\section{Summary} {

		This project aims to detect spheres within an image, based on the shading around its environment, and will have a heavy focus on performance and accuracy. The results of this project will be utilised by the Newcastle Robotics Laboratory and has the potential to impact their results at RoboCup, an international robotic soccer competition. This project is of significant interest to the robot soccer community as the results of this could yield a more efficient method of real-time ball detection, which is also beneficial to those who are interested in ball detection within a competitive sports environment such as basketball or tennis.
		
	}

	\section{Background} {

		Keeping track of a ball is a natural and fundamental aspect of playing soccer, that many human players would not consider to be a skill in itself. Ball tracking is important in robotic soccer, where fast and reliable ball tracking is a challenge that has attracted research in the robot soccer community.

		This project aims to develop methods for sphere detection in a RoboCup environment, placing an emphasis on performance due to the limited capabilities of the target platform. In particular, the performance is significantly effected by the number of pixel samples read from a given image and will be minimised. The project will focus on developing a novel, minimal, multi-stage scanning and ball-detection strategy that minimises the sampling effort while still correctly recognising all balls that are present in the image without false positives.

		An implementation of the developed approach will be created by extending the NUbots’ existing vision system, to minimise the effort required. The results will be compared to the ball detection method used in their current vision system and to other competing ball detection techniques, with a heavy focus on performance and accuracy. If a robust approach is found, it could be generalised to detect other shapes such as robots, lines and goals based on a set of features.

		The following techniques will be considered during research:

		\begin{itemize}
			\item Clustering algorithms
			\begin{itemize}
				\item K-means
				\item DBSCAN
				\item OPTICS
			\end{itemize}
			\item Generic model fitting and improvement algorithms
			\begin{itemize}
				\item Hough transform
				\item Least squares optimisation
			\end{itemize}
			\item Sample consensus techniques
			\begin{itemize}
				\item RANSAC
				\item M-SAC
			\end{itemize}
			\item Machine learning techniques
			\begin{itemize}
				\item SVM
				\item KNN
				\item CNN
			\end{itemize}
			\item Input preprocessing techniques
			\begin{itemize}
				\item Lookup tables
				\item Image derivatives
				\item Filtering and smoothing methods
				\item Scanning strategies
			\end{itemize}
		\end{itemize}

		It is also possible to utilise the information gathered from previous frames, or from the robot platform itself, including the expected position, velocity and kinematics horizon.

	}

	\section{Project plan} {
		
	}


	\section{Project schedule} {

		\begin{figure}[H]
	        \makebox[\textwidth][c]{\resizebox{0.95\paperwidth}{!}{% Define some Gantt chart helper commands:
\newcommand{\completedganttbar}[4][]{ %
    \ganttbar[bar/.append style={draw=gray, fill=gray},#1]{#2}{#3}{#4} %
}
\newcommand{\optionalganttbar}[4][]{ %
    \ganttbar[bar/.append style={draw=gray, pattern color=gray, pattern=north west lines},#1]{#2}{#3}{#4} %
}
\newcommand{\optionalganttlinkedbar}[4][]{ %
    \ganttlinkedbar[bar/.append style={draw=gray, pattern color=gray, pattern=north west lines},#1]{#2}{#3}{#4} %
}

\begin{ganttchart}[
        hgrid,
        vgrid={*6{black, dotted},*1{black, dashed}}, % Note: NO SPACES!
        title height = 1,
        x unit=0.3cm,
        y unit title=0.75cm,
        y unit chart=1cm,
        time slot format=isodate,
        % progress=today,
        % today=2015-8-20,
        % bar/.append style={fill=green},
        % bar incomplete/.append style={fill=white},
        % group incomplete/.append style={draw=black,fill=none},
        % progress label text={}
        ]
        {2015-03-16} % start date: Mon, Mar-16. Week 4.
        {2015-05-31} % end date: Sun, May-31.
\setganttlinklabel{f-s}{}

% Gantt chart header:
% Note: Week starts on Sunday.
\gantttitlecalendar*{2015-03-16}{2015-05-31}{month=name} \\
\gantttitlecalendar*{2015-03-16}{2015-04-05}{week=4}
\gantttitle{Semester 1 Recess}{14}
\gantttitlecalendar*{2015-04-20}{2015-05-31}{week=7}

% Proposal
\ganttnewline \ganttgroup{Project proposal}{2015-03-19}{2015-03-26}
\ganttnewline \ganttbar{Report}{2015-03-19}{2015-03-23}
\ganttnewline \ganttbar{Presentation}{2015-03-24}{2015-03-26}

% Project work
\ganttnewline \ganttgroup{Project work}{2015-03-27}{2015-04-30}
\ganttnewline \ganttbar{Dataset collection}{2015-03-27}{2015-04-05}
\ganttnewline \ganttbar{Candidate selection}{2015-04-01}{2015-04-08}
\ganttnewline \ganttbar{Development}{2015-04-09}{2015-04-22}
\ganttnewline \ganttbar{Report preparation}{2015-04-23}{2015-04-30}

% Final report
\ganttnewline \ganttgroup{Final report}{2015-04-31}{2015-05-28}
\ganttnewline \ganttbar{Detection performance testing}{2015-04-31}{2015-05-07}
\ganttnewline \ganttbar{Analysis of results}{2015-05-08}{2015-05-15}
\ganttnewline \ganttbar{Report preparation}{2015-05-16}{2015-05-28}

% % Examples of gantt chart capabilities: 
% \ganttnewline \ganttgroup{Label Text}{2015-08-20}{2015-10-5}
% \ganttnewline \ganttmilestone{Label Text}{2015-08-3}
% \ganttnewline \completedganttbar{Label Text}{2015-8-9}{2015-8-13}
% \ganttnewline \ganttlinkedmilestone{Label Text}{2015-08-19}
% \ganttnewline \ganttbar{Label Text}{2015-09-11}{2015-10-5}
% \ganttnewline \optionalganttbar{Label Text}{2015-10-6}{2015-10-26}

\end{ganttchart}
}}
			\caption[Project Schedule] {
				A Gantt chart illustrating the planned project schedule. Patterned grey bars represent optional tasks.
			}
			\label{gantt:proposal}
		\end{figure}

	}

	\section{Roles} {
		

	}

	\bibliography{bibliography}
	\bibliographystyle{apalike}

\end{document}

